\documentclass{article}

\usepackage{graphicx}
\usepackage{enumitem}
\usepackage{color}
\usepackage{verbatim}
\usepackage{environ}
\usepackage{mathtools}
\usepackage{hyperref}
\usepackage{amsmath}
\usepackage{listings}
\usepackage{xcolor}


\usepackage{caption}


\renewcommand\lstlistingname{Stata Example}

\usepackage[a4paper, total={6in,9in}]{geometry}

\usepackage{fvrb-ex}

\captionsetup[lstlisting]{name="Stata Output",font={it,tt,color=blue}}

\usepackage{eurosym}
\usepackage{amstext} % for \text
\DeclareRobustCommand{\officialeuro}{%
  \ifmmode\expandafter\text\fi
  {\fontencoding{U}\fontfamily{eurosym}\selectfont e}}


\fvset{gobble=0,numbersep=3pt}
\fvset{numbers=left,frame=single}
%\RecustomVerbatimEnvironment{Verbatim}{Verbatim}{commandchars=§µ¶}
\DefineVerbatimEnvironment%
{CVerbatim}{Verbatim}
{fontfamily=tt,fontsize=\small,frame=single,formatcom=\color{blue},label=\emph{Stata code/output}}
\DefineVerbatimEnvironment{Sinput}{Verbatim}{fontshape=sl,formatcom=\color{blue}}

\lstset{ %
  basicstyle=\footnotesize\ttfamily\color{blue}, %
  frame=single, %
  includerangemarker = false, %
  title = \footnotesize\ttfamily\color{blue}\emph{Interactive Stata Example}, %
  caption = \footnotesize\ttfamily\color{blue}\emph{}, %
  captionpos = t, %
  rangeprefix=*@*, %
  %belowcaptionskip = -0.08in, %
  label={bla}
}



\begin{document}

\title{\small Applied Panel Econometrics (MPP-E1161) - Winter Term 2015 \\ Prof. Dr. Kerstin Bernoth \\ \bigskip \Large Take Home Exam 2}
\author{Max Callaghan}
\date{November 2015}
\maketitle

\section{Part I: Basic Questions [12pt: each 2pt]}
Briefly explain why your chosen answer is correct.
\begin{enumerate}
  \item
  \begin{description}
    \item[Question] \hfill \\
    False or true: When the between group variance of a panel data set is small, the random effects estimator looks similar to the fixed-effects estimator.
    \item[Answer] \hfill \\
    The fixed-effects estimator discards information between panels and only concentrates on variation within panels. The random-effects estimator is different to the fixed-effects estimator because it takes into account between-group variation as well as within group variation; if between group variance is small, then the difference betweeen FE and RE will also be small. 
    
  \end{description}
  
  \item
  \begin{description}
    \item[Question] \hfill \\
    False or true: A variable \(z\) serves as a good instrument for an endogenous explanatory variable \(x\), if it is sufficiently correlated with the dependent variable \(y\)
    \item[Answer] \hfill \\
    False. A good instrument \(z\) is one that is correlated with \(x\) but uncorrelated with the error term \(\epsilon\)
    
  \end{description}
  
  \item
  \begin{description}
    \item[Question] \hfill \\
    False or true: The Hausman test tests, whether the estimated coefficients of two regressions are not significantly different.
    \item[Answer] \hfill \\
    True. The Hausman test shows the probability of there being no significant difference between the coefficients of two regressions.
  \end{description}
  
  \item
  \begin{description}
    \item[Question] \hfill \\
    False or true: If the fixed effects and the random effects estimator deliver significantly different coefficients, we should prefer to use the fixed-effects estimator.
    \item[Answer] \hfill \\
    If individual effects are correlated with explanatory variables, we violate the assumption of the exogeneity of regressors, rendering our estimate inconsistent and biased. A fixed-effects model controls for individual effects and so will be unbiased and consistent. So if the difference between the estimators of FE and RE models are significantly large, we can assume that exogeneity of regressors is violated and that RE estimators are therefore biased and inconsistent.
    
  \end{description}
  
  \item
  \begin{description}
    \item[Question] \hfill \\
    False or true: The larger the correlation between the endogenous variable \(x\) and its instrument \(z\), the less precise is the instrumental variable estimator.
    \item[Answer] \hfill \\
    False. The variance of the regressor is given by
    
    \[ \hat{Var}(\hat{\beta}^{IV}_{1}) = \frac{\hat{\sigma}^2_{\epsilon}\frac{1}{NT}}{Var(x_{it})\rho^2_{x,z}},  \]
    
    where \(\rho^2_{x,z}\) is the square of the correlation between \( x \) and \( y \).
    
    So an increase in the correlation between x and z will decrease the variance of the IV estimator.
  \end{description}
  
  \item
  \begin{description}
    \item[Question] \hfill \\
    Even if the single parameter \(t\)-test suggests that each coefficient is insignificant, the \(F\)-test might say that these coefficients are jointly significant.
    \item[Answer] \hfill \\
    True. We might have two instruments that suffer from multicollinearity. In this case, though the two are jointly significant, each would make the other insignificant.
  \end{description}
\end{enumerate}
  
\section{Part II: Model Interpretation [9pt: each 3pt]}

\begin{enumerate}
  \item
  \begin{description}
    \item[Question] \hfill \\
    \[spread_{it} = 21 + 0.2debt_{it} + 1.3deficit_{it} + 0.05debt_{it} \cdot deficit_{it} + \epsilon_{it}\]
    \item[Answer] \hfill \\
    \begin{itemize}
      \item A country with a debt of 0 and a deficit of 0 is expected have a bond yield spread of 21.
      \item An increase in the deficit by 1 percentage point increases the country's bond yield spread by \(1.3 + 0.05 \cdot deficit \) 
      \item Thus, for a country with a debt of 20 percentage points, every percentage point increase in the deficit is expected to increase the bond yield spread by 2.3
    \end{itemize}
  \end{description}
  
  \item
  \begin{description}
    \item[Question] \hfill \\
    \[ spread_{it} = 13 + 0.15debt_{it} + 23crisis + 0.3debt \cdot crisis + \epsilon_{it} \]
    \item[Answer] \hfill \\
    \begin{itemize}
      \item A country without any debt is expected to have a bond yield spread of 13 outside of a crisis and \( (13+23)=36 \) during a crisis.
      \item Outside of a crisis, every percentage point increase in debt is expected to increase bond yield spread by 0.15 
      \item During a crisis, every percentage point increase in debt is expected to increase bond yield spread by \( (0.15 + 0.3) = 0.45 \) 
    \end{itemize}
  \end{description}
  
  \item
  \begin{description}
    \item[Question] \hfill \\
    \[ spread_{it} = 2.1debt_{it} - 0.01debt_{it}^2 + \epsilon_{it} \]
    \item[Answer] \hfill \\
    \begin{itemize}
      \item Debt levels have a diminishing effect on bond yield spreads.
      \item The effect of debt levels on bond yield spreads follows an inverted U-shape. At low levels it has positive marginal effects, but at higher levels it has negative marginal effects.
      \item The marginal effect of debt levels on bond yield spreads is 
      \[ \frac{\partial Spread}{\partial Debt} = 2.1 - 2 \cdot 0.01 \]
      \item The  highest bond yield spreads are estimated at 
      \[ debt* = \frac{2.1}{2\cdot 0.01} = 105 \]
      \item When \(debt < 105\), bond yield spreads increase with every additional percentage point of debt. When \(debt > 105\), bond yield spreads decrease with every additional percentage point of debt
    \end{itemize}
  \end{description}
  
\end{enumerate}

\section{Part 3: OLS and IV regression [20pt]}

\[spread_{it} = \alpha_i + \beta_{1}deficit + u_{it} \]

\begin{enumerate}
  \item
  \begin{description}
    \item[Question] \hfill \\
    Write down the formula for \(\hat{\beta}^{FE}_1\) and \(\hat{\alpha}_i\) 
    \item[Answer] \hfill \\
    \[ Y_{it} = \beta_{1}X_{it} + \alpha_{i} + u_{it} \]
    \[ y^{*}_{it} = x^{*}_{it}\beta + u^{*}_it, \]
    
    where 
    
    \null
    
    \(y^{*}_{it} = y_{it} - \bar{y}_{i} \) and \(x^{*}_{it} = x_{it} - \bar{x}_{i}, \)
    
    \null 
    so 
    
    \[\hat{\beta}^{FE}_1 = \frac{cov(x_{it},y_{it})}{var(x_{it})} \] and
    
    \[ \hat{\alpha}_i = \bar{y}_{i} - \bar{x}_{i}\hat{\beta} \]
    
  \end{description}
  
  \item
  \begin{description}
    \item[Question] \hfill \\
    Calculate \(\hat{\beta}^{FE}_1\) and \(\hat{\alpha}_i\) and write down the regression equation for all three countries. (10pt)
    \item[Answer] \hfill \\
    
    \textbf{Ireland:} \( Spread_{t} = 617.4 -66(Deficit_{t}) + \epsilon \)
    
    \textbf{Netherlands:} \( Spread_{t} = 76.6 -66(Deficit_{t}) + \epsilon \)
    
    \textbf{Spain:} \( Spread_{t} = 272.2 -66(Deficit_{t}) + \epsilon \)
    
  \end{description}
  
  \item
  \begin{description}
    \item[Question] \hfill \\
    We assume that the variable \(deficit\) is endogenous and we want to estimate regression (1) with an instrumental variable estimation, where lagged deficit (\(L.deficit\)) serves as an instrument for \(deficit\). Explain, why \(L.deficit\) might be a suitable instrument for deficit. (4pt)
    \item[Answer] \hfill \\
    We have an endogeneity problem because a high bond yield spreads push up the cost of borrowing and therefore increase the deficit. An instrumental variable must be exogenous and informative or relevant. Lagged deficit should correlated with the current deficit (relevant), but should be independent of the error term (exogenous).
    
  \end{description}
  
  \item
  \begin{description}
    \item[Question] \hfill \\
    Add \(z_{it} = L.deficit_{it} \) in the empty column in the Table. It might happen that you have missing observations. (2p5)
    \item[Answer] \hfill \\
    
  \end{description}
  
  \item
  \begin{description}
    \item[Question] \hfill \\
    Write the IV formula for \(\hat{\beta}^{IV}_1\) and \(\hat{\alpha}^{IV}_i\)
    \item[Answer] \hfill \\
    
    \[ \hat{\beta}^{IV}_1 = \frac{Cov(z_{it},y_{it})}{Cov(z_{it},x_{it})} \textnormal{ and } \hat{\alpha}^{IV}_i = \bar{y}_{i} - \bar{x}_{i}\hat{\beta} \]
    
  \end{description}
  
  \item
  \begin{description}
    \item[Question] \hfill \\
    Calculate (not estimate) \(\hat{\beta}^{IV}_1\) and \(\hat{\alpha}^{IV}_i\) and write down the IV regression equation for all three countries. You may extend the table with as many columns as necessary. Write down all calculations (Covariances, Variances, etc.) that are necessary. (10pt)
    \item[Answer] \hfill \\
    
    \textbf{Ireland:} \( Spread_{t} = 594.3 -62.5(Deficit_{t}) + \epsilon \)
    
    \textbf{Netherlands:} \( Spread_{t} = 74.5 -62.5(Deficit_{t}) + \epsilon \)
    
    \textbf{Spain:} \( Spread_{t} = 261.7 -62.5(Deficit_{t}) + \epsilon \)
    
  \end{description}
  
\end{enumerate}

\section{Part III: Current Account Imbalances and Exchange Rate Regimes - Continue [37pt]}

\begin{enumerate}
  \item \textbf{Deriving the model specification}
  \begin{enumerate}[label=(\alph*)]
    \item 
    \begin{description}
      \item[Question] \hfill \\
      Explore, whether you prefer pooled OLS, the fixed- or the random effects estimation. Explain, how you have derived your conclusion. [4pt]
      \item[Answer] \hfill \\
      \begin{figure}
      \lstinputlisting[linerange={lstart-lend},caption=Testing for Heteroskedasticity,label=q1_1]{../stata/exam_2_q3_1_a_1.log}
      \end{figure}
      
      A Breusch-Pagan LM test found that the probability of constant variance was close to 0, meaning that we should be wary of pooled OLS and consider panel techniques (see Stata Output ~\ref{q1_1})
      
      \begin{figure}
      \lstinputlisting[linerange={lstart-lend},caption=Fixed-Effects: F-test,label=q1_3]{../stata/exam_2_q3_1_a_2.log}
      \end{figure}
      
      After running a fixed-effects model, the F test makes it clear that FE is more efficient than pooled OLS (see Stata Output ~\ref{q1_2}).
      
      \begin{figure}
      \lstinputlisting[linerange={lstart-lend},caption=Comparing Fixed and Random Effects,label=q1_3]{../stata/exam_2_q3_1_a_3.log}
      \end{figure}
      
      The Hausman test (see see Stata Output ~\ref{q1_3}) finds that the estimated coefficients of FE and RE do not differ significantly. We should therefore prefer a random-effects model, which will be unbiased and more efficient.
      
    \end{description}
    \item 
    \begin{description}
      \item[Question] \hfill \\
      Test for the presence of serial correlation, cross-sectional dependence and panel heteroscedasticity. [3pt]
      \item[Answer] \hfill \\
      \begin{itemize}
        \item Because N (59) is greater than T (35) we use the cross-sectional dependence (CD) test
        \item Pesaran and Frees abundantly reject the hypothesis that there is no cross sectional dependence see~\ref{q1_b_cd}. Pesaran is optimised for unbalanced panels, which our dataset is, so we must accept that there is cross sectional dependence.
      \begin{figure}
      \lstinputlisting[linerange={lstart-lend},caption=Cross Sectional Dependence,label=q1_b_cd]{../stata/exam_2_q3_1_b_cd.log}
      \end{figure}
      \item Woolridge
      \begin{figure}
      \lstinputlisting[linerange={lstart-lend},caption=Cross Sectional Dependence,label=q1_b_sc]{../stata/exam_2_q3_1_b_sc.log}
      \end{figure}
      \end{itemize}
    \end{description}
    \item 
    \begin{description}
      \item[Question] \hfill \\
      Explain, why we need to take the residual structure into account. [2pt]
      \item[Answer] \hfill \\
      Both RE and FE models assume that the the presence of \(\alpha_i\) captures all correlation between the unobservables in different time periods, and so \(u_{it}\) is assumed to be uncorrelated over individuals over time and i.i.d. distributed.
      
      If this is not the case then the estimated coefficients will still be unbiased and consistent but the standard errors and resulting tests will be innacurate, and the estimators are no longer efficient.
    \end{description}
    \item 
    \begin{description}
      \item[Question] \hfill \\
      Estimate your preferred model (pooled OLS, fixed- or random effects) taking the residual structure into account to get unbiased and efficient results. Explain the choice of your estimator. [2pt]
      \item[Answer] \hfill \\
      \begin{figure}
      \lstinputlisting[linerange={lstart-lend},caption=Estimating with Panel Corrected Standard Errors,label=q3_1_d]{../stata/exam_2_q3_1_d.log}
      \end{figure}
      Stata Output~\ref{q3_1_d} shows a model with panel-corrected standard errors. They are corrected for panel heteroscedasticity, autocorrelation and contemporaneously cross-sectional correlation, all which are present in the data.
    \end{description}
    \item 
    \begin{description}
      \item[Question] \hfill \\
      Indicate, whether the Friedman Hypothesis holds. [1pt]
      \item[Answer] \hfill \\
      The Friedman Hypothesis still does not hold. The effect of regime (though it is problematic to treat it as a continuous variable as the distance between 1 and 2 is not the same as the distance between 2 and 3) is now positive (supporting Friedman's Hypothesis), but insignificant.
    \end{description}
  \end{enumerate}
  \item \textbf{Estimation with interaction variables}
  \begin{enumerate}[label=(\alph*)]
    \item 
    \begin{description}
      \item[Question] \hfill \\
      Generate an interaction variable between the variable \(regime\) and the dummy \(id\) and repeat your regression by adding this interaction variable together with the dummy \(id\). [2pt]
      \item[Answer] \hfill \\
      Stata output~\ref{q2_a} shows the results of a regression including all previous regressors, the dummy variable \(id\) (describing if a country is industrialised or not) and an interaction \(reg_x_id\) of id and regime.
      \begin{figure}
      \lstinputlisting[linerange={lstart-lend},caption=Introducing an Interaction term,label=q2_a]{../stata/exam_2_q3_2_a.log}
      \end{figure}
    \end{description}
    \item 
    \begin{description}
      \item[Question] \hfill \\
      Hypothesis testing: Test, whether in case of industrial countries the exchange rate regime affects current account imbalances.
      \item[Answer] \hfill \\
    \end{description}
    \item 
    \begin{description}
      \item[Question] \hfill \\
      Give a numerical interpretation of the effect of the exchange rate regime on current account imbalances for industrial and non-industrial countries [2pt].
      \item[Answer] \hfill \\
    \end{description}
    \item 
    \begin{description}
      \item[Question] \hfill \\
      Based on your estimation results, what would you recommend policymakers, when you are asked about the preferred exchange rate regime.
      \item[Answer] \hfill \\
    \end{description}
  \end{enumerate}
\end{enumerate}
    
\end{document}